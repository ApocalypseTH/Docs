\section{Introduzione}

\subsection{Luogo e data dell'incontro}
	\begin{itemize}
		\item \textbf{luogo:} Discord
		\item \textbf{data:} 2021-01-05
		\item \textbf{ora di inizio:} 19:00
		\item \textbf{ora di fine:} 20:15
	\end{itemize}

\subsection{Segretario}
Il segretario del verbale è \textbf{Giovanni Michieletto}

\subsection{Presenze}
	\begin{itemize}
		\item \textbf{totale presenti:} 7
		\item \textbf{presenti: }
			\begin{itemize}		
				\item Riccardo Calcagno
				\item Giovanni Michieletto
				\item Samuele Sartor
				\item Vittorio Schiavon
				\item Alberto Sinigaglia
				\item Marco Tesser
				\item Alice Zago
			\end{itemize}
		\item \textbf{assenti: } 0
	\end{itemize}


\newpage
\section{Resoconto}
Nel meeting son stati trattati i seguenti argomenti
	\begin{itemize}
		\item\textbf{Conferma modello documenti di LaTeX}:
		Si è decisa e concordata la struttura finale del modello LaTex su cui si baseranno i documenti e i verbali. In particolare si è deciso di adottare all’interno dei verbali una tabella riassuntiva delle decisioni in modo da identificarle e categorizzarle. Si è deciso inoltre di iniziare la trascrizione di tutti i documenti da Google Docs a LaTeX.
		\item\textbf{Discussione delle Norme di Progetto}:
			Si sono discusse le norme di progetto mancanti, o non ancora pienamente definite, in modo da completarne la stesura e rendere tutti i documenti in linea con esse. Sono state decise le tecnologie da utilizzare per la comunicazione, Discord e Telegram, e le tecnologie utilizzate per la stesura dei documenti, ovvero Google sheet, TexStudio/TexShop e StarUml. 
			In particolare sono stati definiti o aggiornati i seguenti capitoli delle norme di progetto:
			\begin{itemize}
				\item Tipologie di documenti prodotti 
				\item Ciclo di vita del ticket 
				\item Tecnologie e linguaggi per formazione
			\end{itemize}
		\item\textbf{ Modifica del modello del verbale}:
		Si è discussa la struttura del modello dei verbali e si è deciso di modificarla, in particolare si è deciso di rimuovere l’ordine del giorno e di aggiungere una tabella di tracciamento delle decisioni nella quale ogni decisione avrà il proprio codice identificativo. 
		\item\textbf{Discussione dell’Analisi dei Requisiti}:
		È stato visionato il documento dell’Analisi dei Requisiti allo stato attuale e sono stati discussi i vari problemi riscontrati riguardo l’identificazione e la distinzioni dei requisiti, tra essenziali e di design. A tal proposito si è deciso di chiedere chiarimenti al proponente.
		\item\textbf{Confronto domande da porre al proponente}:
		Si sono decise e concordate le domande da porre al proponente durante il meeting dell’8 Gennaio 2021 in modo da sfruttare a pieno il tempo a disposizione evitando ripetizioni e sprechi di tempo.
	\end {itemize}