\section{Introduzione}

\subsection{Luogo e data dell'incontro}
	\begin{itemize}
		\item \textbf{luogo:} Discord
		\item \textbf{data:} 2020-12-17
		\item \textbf{ora di inizio:} 17:15
		\item \textbf{ora di fine:} 18:15
	\end{itemize}

\subsection{Segretario}
Il segretario del verbale è \textbf{Alberto Sinigaglia}

\subsection{Presenze}
	\begin{itemize}
		\item \textbf{totale presenti:} 7
		\item \textbf{presenti: }
			\begin{itemize}		
				\item Riccardo Calcagno
				\item Giovanni Michieletto
				\item Samuele Sartor
				\item Vittorio Schiavon
				\item Alberto Sinigaglia
				\item Marco Tesser
				\item Alice Zago
			\end{itemize}
		\item \textbf{assenti: } 0
	\end{itemize}


\newpage
\section{Resoconto}
Nel meeting son stati trattati i seguenti argomenti
	\begin{itemize}
		\item\textbf{Decisione workflow di integrazione e contribuzione alla documentazione}:
		Si è deciso che la contribuzione alla documentazione avverrà tramite una repository Github, tramite l’uso di Pull Request, le quali dovranno essere proposte dai vari componenti, e verificate da componenti diversi dal contributore, prima di essere accettate
		\item\textbf{Definizione struttura generale della repository su Github per la documentazione}:
		Si è deciso che la repository Github contenente la documentazione, dovrà aver al suo interno una cartella con i template da usare per i vari tipi di documenti quali:
		\begin{itemize}
			\item documenti
			\item verbali
			\item lettere
		\end{itemize}
		Tali template sarà obbligatorio usarli sia per la stesura che per la contribuzione alla documentazione

		\item\textbf{Definizione di chi dovrà fare una prima analisi e primo sviluppo dei vari documenti}:
		È stato creato un indirizzo email con il nome del gruppo per la gestione del progetto.
		Si è deciso che ogni componente del gruppo avrà in carico l’analisi di uno dei documenti di progetto, così da parallelizzare il lavoro, che saranno poi da integrare con i template LaTeX forniti sulla repository principale.
		I ruoli sono stati così definiti:
		\begin{itemize}
			\item norme di progetto: Riccardo Calcagno e Alice Zago
			\item analisi dei requisiti: Marco Tesser e Vittorio Schiavon
			\item piano di progetto: Giovanni Michieletto e Samuele Sartor
			\item piano di qualifica: Alberto Sinigaglia
		\end{itemize}
		Si è inoltre deciso che nel caso ci fossero conclusioni anticipate della stesura dei documenti, i relativi redattori dovranno andare a supportare i rimanenti

	\end{itemize}

\subsection{Prossima riunione}
La prossima riunione è stata fissata come segue:
\begin{itemize}
	\item \textbf{luogo:} chiamata su Discord
	\item \textbf{data:} 2021-01-05
	\item \textbf{ora di inizio:} 19:00
\end{itemize}


