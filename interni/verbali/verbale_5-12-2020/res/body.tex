\section{Introduzione}

\subsection{Luogo e data dell'incontro}
	\begin{itemize}
		\item \textbf{luogo:} Discord
		\item \textbf{data:} 2020-12-05
		\item \textbf{ora di inizio:} 10:00
		\item \textbf{ora di fine:} 11:30
	\end{itemize}

\subsection{Segretario}
Il segretario del verbale è \textbf{Marco Tesser , Samuel Sartor}

\subsection{Presenze}
	\begin{itemize}
		\item \textbf{totale presenti:} 7
		\item \textbf{presenti: }
			\begin{itemize}		
				\item Riccardo Calcagno
				\item Giovanni Michieletto
				\item Samuele Sartor
				\item Vittorio Schiavon
				\item Alberto Sinigaglia
				\item Marco Tesser
				\item Alice Zago
			\end{itemize}
		\item \textbf{assenti: } 0
	\end{itemize}


\newpage
\section{Resoconto}
Nel meeting son stati trattati i seguenti argomenti
	\begin{itemize}
		\item\textbf{Nome del gruppo}:
		A seguito di una breve discussione di gruppo è stato deciso tramite votazioni di dare al gruppo il nome EverBuilds.
		\item\textbf{Logo del gruppo}:
		È stato deciso a chi assegnare il compito di creare il logo del gruppo e successivamente creato.
		\textbf{Creatori}: Vittorio Schiavon e Giovanni Michieletto
		\item\textbf{ Indirizzo email}:
		È stato creato un indirizzo email con il nome del gruppo per la gestione del progetto.
		\item\textbf{ Strutturazione su google drive per documentazione }:
		È stato strutturato su google drive lo schema delle cartelle da utilizzare per il progetto, e con esso il primo documento: Verbale\_Interno\_5-12-2020.
		\item\textbf{ Repository github }:
		È stato creato il repository Github per la gestione del progetto.
		Repository: \url{https://github.com/everbuilds/ProgettoSWE_everbuilds}
	\end{itemize}

\subsection{Prossima riunione}
La prossima riunione è stata fissata come segue:
\begin{itemize}
	\item \textbf{luogo:} chiamata su Discord
	\item \textbf{data:} 2020-12-12
	\item \textbf{ora di inizio:} 10:30
\end{itemize}


\subsection{Varie ed eventuali}
\begin{itemize}

	\item è stata consigliata una breve guida per \LaTeX{} su internet;
	\item entro la prossima settimana verrà eseguita una prima configurazione della \glock{repository}.
\end{itemize}


