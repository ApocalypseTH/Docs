\section{Introduzione}

\subsection{Luogo e data dell'incontro}
	\begin{itemize}
		\item \textbf{luogo:} Discord
		\item \textbf{data:} 2020-12-12
		\item \textbf{ora di inizio:} 10:30
		\item \textbf{ora di fine:} 12:00
	\end{itemize}

\subsection{Segretario}
Il segretario del verbale è \textbf{Marco Tesser , Samuel Sartor}

\subsection{Presenze}
	\begin{itemize}
		\item \textbf{totale presenti:} 7
		\item \textbf{presenti: }
			\begin{itemize}		
				\item Riccardo Calcagno
				\item Giovanni Michieletto
				\item Samuele Sartor
				\item Vittorio Schiavon
				\item Alberto Sinigaglia
				\item Marco Tesser
				\item Alice Zago
			\end{itemize}
		\item \textbf{assenti: } 0
	\end{itemize}


\newpage
\section{Resoconto}
Nel meeting son stati trattati i seguenti argomenti
	\begin{itemize}
		\item\textbf{Verifica studio di fattibilità}:
		È stato deciso di assegnare ad ogni componente del gruppo due degli studi fattibilità redatti per questa settimana dai compagni per essere verificati. Abbiamo ritenuto che le informazioni proposte dovessero essere approvate dalla totalità del gruppo perché il documento soddisfa la necessità di dare voce ad ognuno dei singoli membri.
		\item\textbf{Strutturazione norme di progetto}:
		È stata decisa la lista dei punti da inserire nelle norme di progetto e strutturato il relativo indice.
		\item\textbf{Integrazione Latex}:
		Si è scelta la strategia per integrare l’uso di Latex nella documentazione standardizzata che verrà caricata su Github.
		\item\textbf{Discussione linguaggi programmazione}:
		Sono stati discussi i linguaggi che verranno utilizzati nel capitolato e stilata la lista sui quali doversi auto-formare per la realizzazione del capitolato.
		Confrontati Kotlin e Swift, confermato node.js. 
	\end{itemize}

\subsection{Prossima riunione}
La prossima riunione è stata fissata come segue:
\begin{itemize}
	\item \textbf{luogo:} chiamata su Discord
	\item \textbf{data:} 2020-12-17
	\item \textbf{ora di inizio:} 17:00
\end{itemize}



