\section{Valutazione capitolato scelto}
	\subsection{Capitolato C6 - RGP: Realtime Gaming Platform}
		\subsubsection{Informazioni generali}
			\begin{itemize}
				\item\textbf{Nome}: Realtime Gaming Platform
				\item\textbf{Proponente}: Zero12 s.r.l.
				\item\textbf{Committente}: Prof. Tullio Vardanega e Prof. Riccardo Cardin
			\end{itemize}
			
		\subsubsection{Descrizione del capitolato}
			Il progetto proposto prevede la realizzazione di un videogioco a scorrimento verticale fruibile da device mobile (iOS o Android) con la possibilità di giocare sia in singleplayer che in multiplayer in realtime
			È proprio la sfida tra i giocatori a rappresentare il fulcro del progetto, in particolare sarà richiesta una sfida ad eliminazione dove l’ultimo giocatore eliminato vince. 
			Diversamente, una sessione condotta in singleplayer sarà in modalità infinita con livelli crescenti di difficoltà, e un limitato numero di vite.
		
		\subsubsection{Finalità del progetto}
			I requisiti di progettazione dell’applicazione sono i seguenti :
			\begin{itemize}
				\item Sviluppare un architettura server cloud-based che permetterà di gestire la comunicazione tra device (da 2 a 6 giocatori per sessione).
				\item La sfida per ogni giocatore deve risultare la medesima quindi sarà necessario sincronizzare il comportamento di eventuali nemici e powerup. 
				\item Non è prevista l’interazione tra giocatori ma durante la partita deve essere comunque possibile vedere i movimenti degli avversari in modalità “fantasma”.
				\item L’architettura server dovrà essere scalabile e strutturata a microservizi.
			\end{itemize}
			 I passi essenziali per portare a termine il progetto richiesti dai redattori del capitolato sono i seguenti:
			\begin{itemize}
				\item Scouting delle tecnologie AWS per capire quale può adattarsi meglio ad un gioco con requisiti realtime, motivando e documentando la scelta.
				\item Implementazione della componente server-side.
				\item Implementazione del gioco per piattaforma mobile.
			\end{itemize}
			Se è vero che ci sono vincoli e riferimenti da seguire sulle dinamiche del gameplay e interazioni dei giocatori siamo invece liberi di scegliere lo stile estetico del gioco
		\subsubsection{Tecnologie interessate}
			Il software richiesto dovrà utilizzare le seguenti tecnologie:
			\begin{itemize}
				\item\textbf{AWS}: Amazon Web Services tecnologia necessaria a realizzare l’architettura server cloud-based per la gestione dei giochi in realtime. Alcuni esempi proposti di questa tecnologia tra cui scegliere sono: \textbf{AWS GameLift} e \textbf{AWS Appsync}.
				\item\textbf{NodeJS}: tecnologia preferibile se viene scelto un servizio che richiede lo sviluppo di codice.
				\item\textbf{Swift/SwiftUI e Kotlin}:  linguaggi specifici per lo sviluppo mobile relativi rispettivamente agli ambienti: iOS (minimo iOS 13) e Android (minimo Android 8), è sufficiente fornire una soluzione basata solo su uno dei due ambienti (preferenza: iOS).
			\end{itemize}
			
		\subsubsection{Aspetti positivi}
			\begin{itemize}
				\item L’azienda si è rivelata molto disponibile a chiarimenti futuri e seminari. Inoltre essendo pratica di un approccio agile si è ritenuto che essa possa avere competenze comunicative già integrate nel loro way of working.
				\item Il documento e la presentazione che hanno fatto del capitolato sono stati chiari e consistenti. La fattibilità del progetto è stata percepita nitidamente da tutti i componenti del gruppo di lavoro.
				\item I requisiti espressi dal capitolato lasciano notevole spazio per decisioni e future rielaborazioni da parte del nostro gruppo, come il design del gioco le tecnologie utilizzabili. Ciò ci consentirà di responsabilizzarci maggiormente e affinare un buon senso critico.
				\item La presenza di una forte componente creativa nella realizzazione del gioco ha suscitato interesse in molti membri del gruppo.
				\item È un progetto che ci consentirà di formarci sullo sviluppo nativo di applicazioni per piattaforme mobile di consolidata popolarità come iOS e Android.
				\item Il progetto ci metterà nelle condizioni di apprendere e utilizzare le più recenti tecnologie di sviluppo cloud-based proposte da Amazon Web Services.
			\end{itemize}
			
		\subsubsection{Criticità e fattori di rischio}
			\begin{itemize}
				\item Buona parte del gruppo di lavoro non conosce i linguaggi di programmazione richiesti per lo sviluppo mobile, ciò necessita di un periodo di apprendimento più lungo.
			\end{itemize}
\newpage
				
				
				
				
				
				
				
				
				
				
				
				
				
				
				
				
				
				
				
				
				
				
				
				
				
				
				
				
				
				
				