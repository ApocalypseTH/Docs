\section{Introduzione}
    \subsection{Scopo del documento}
        Questo documento ha lo scopo di definire le regole che tutti i membri del gruppo EverBuilds devono rispettare durante lo svolgimento del progetto affinchè si possa ottenere uniformità in tutti i file prodotti.
        I membri del gruppo sono quindi obbligati a prendere visione di tale documento e rispettare le norme in esso descritte.
        Il documento verrà redatto seguendo una filosofia \textbf{incrementale}, in modo da poter normare passo dopo passo ogni decisione discussa e concordata tra i membri del gruppo. Le norme già presenti potranno inoltre essere sottoposte a cambiamenti (aggiunte, rimozioni o modifiche).
    \subsection{Scopo del prodotto}
        Il \glock{Capitolato} C6 ha come obiettivo quello di realizzare un videogioco a scorrimento verticale fruibile da device mobile con la possibilità di giocare sia in singleplayer sia in \glock{Multiplayer} in real time. 
    \subsection{Glossario}
        Al fine di evitare ambiguità o incongruenze relative al linguaggio utilizzato nei documenti formali, e di conseguenza anche in questo documento, viene fornito un glossario. Tale struttura informativa è individuabile nel documento \dext{Glossario} v1.0.0, contenente i termini che potrebbero far insorgere incomprensioni e la loro spiegazione.
    \subsection{Riferimenti}
        \subsubsection{Riferimenti normativi}
            \begin{itemize}
                \item\textbf{Capitolato d'appalto C6} : \\
                \href{https://sesaspa-my.sharepoint.com/personal/s_dindo_vargroup_it/_layouts/15/onedrive.aspx?id=\%2Fpersonal\%2Fs\%5Fdindo\%5Fvargroup\%5Fit\%2FDocuments\%2FDownload\%2Fupload\%2FIngegneria\%5Fsoftware\%2FCapitolato\%5FIngegneria\%5Fsoftware\%2Epdf&parent=\%2Fpersonal\%2Fs\%5Fdindo\%5Fvargroup\%5Fit\%2FDocuments\%2FDownload\%2Fupload\%2FIngegneria\%5Fsoftware&originalPath=aHR0cHM6Ly9zZXNhc3BhLW15LnNoYXJlcG9pbnQuY29tLzpiOi9nL3BlcnNvbmFsL3NfZGluZG9fdmFyZ3JvdXBfaXQvRVRodmF5MGY2S1ZDb1h5ZFlPY2UybGtCdC1NWWNuVzF5YWZSWEZYVklPSXNIZz9ydGltZT13T1VuQ0F5czJFZw }{https://sesaspa-my.sharepoint.com/}
            \end{itemize}
        \subsubsection{Riferimenti informativi}
            \begin{itemize}
                \item\textbf{Standard ISO/IEC 12207:1995} : \\
                \href{https://www.math.unipd.it/~tullio/IS-1/2009/Approfondimenti/ISO\_12207-1995.pdf}{https://www.math.unipd.it/~tullio/IS-1/2009/Approfondimenti/ISO\_12207-1995.pdf}
                \item\textbf{Documentazione LaTeX} : \\
                \href{ https://www.latex-project.org/help/documentation/}{ https://www.latex-project.org/help/documentation/}
                \item\textbf{Documentazione GitHub} : \\
                \href{ https://help.github.com/en/github}{ https://help.github.com/en/github}
                \item\textbf{Documentazione Git} : \\
                \href{ https://git-scm.com/docs}{ https://git-scm.com/docs}
                \item\textbf{Cost Variance}: \\
                \href{http://acqnotes.com/acqnote/tasks/cost-variances}{http://acqnotes.com/acqnote/tasks/cost-variances}
                \item\textbf{Schedule Variance}: \\
                \href{http://acqnotes.com/acqnote/tasks/schedule-variances}{http://acqnotes.com/acqnote/tasks/schedule-variances}
            \end{itemize}
\newpage
        
        
        
        
        
        
        
        
        
        
        
        
        
        
        
        
        
        
        
        
        
        
        