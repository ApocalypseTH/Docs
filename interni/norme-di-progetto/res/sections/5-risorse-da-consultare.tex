\section{Risorse da consultare}
    \subsection{Liste di controllo}
        Per svolgere il procedimento dell’ \glock{inspection} della documentazione con \glock{Efficienza} ed \glock{Efficacia} i verificatori si devono avvalere delle liste di controllo riportate in questa sezione, sono differenziate sulla base del tipo di proprietà da verificare, tutte le liste di controllo sono aperte a modifiche e si invitano i verificatori a integrare le loro conoscenze sugli errori ricorrenti.\\
        \subsubsection{Liste di controllo documentazione}
            \paragraph{Norme stilistiche e correttezza linguistica}
                \begin{itemize}
                    \item\textbf{elenchi puntati e numerati}: non terminano con il punto e virgola se non si tratta dell’ultimo elemento oppure non terminano con il punto se si tratta dell’ultimo elemento;
                    \item\textbf{Gli elementi soggetto di un elemento di un elenco}: non sono rappresentati in grassetto;
                    \item\textbf{Didascalie immagini e tabelle}: non presenti o non rispettano il formato descritto dalle norme;
                    \item\textbf{Ruoli di progetto, nomi di documenti e nomi di processi}: non sono scritti in corsivo;
                    \item\textbf{Nomi dei documenti}: I nomi dei documenti non rispettano il formato descritto dalle norme;
                    \item\textbf{Virgole e punti}: utilizzati o troppo frequentemente o troppo poco frequentemente complicando la struttura della frase;
                    \item\textbf{G di Glossario}: Una parola riportata in \dext{Glossario} v1.0.0 non è stata caratterizzata dall’aggiunta a pedice di un G maiuscola;
                    \item\textbf{Esempio}: non è preceduto dalla formula: “e.g.”;
                \end{itemize}
                
        \subsubsection{Liste di controllo codice}
            qui verranno aggiunte incrementalmente i punti della lista di controllo per verificare il codice per \glock{inspection}.
            
