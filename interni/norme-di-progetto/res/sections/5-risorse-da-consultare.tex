\section{Risorse da consultare}
	\subsection{Liste di controllo}
		Per svolgere il procedimento dell’ inspection della documentazione con efficienza ed efficacia i verificatori si devono avvalere delle liste di controllo riportate in questa sezione, sono differenziate sulla base del tipo di proprietà da verificare, tutte le liste di controllo sono aperte a modifiche e si invitano i verificatori a integrare le loro conoscenze sugli errori ricorrenti.\\
		Ogni punto delle liste di controllo è decorato con un carattere a pedice che ne identifica la tipologia dell’atto da compiere per verificarlo:\\
		\begin{itemize}
			\item\textbf{‘e‘ a pedice dell’elemento}: Identifica un elemento della lista che può essere verificato puntualmente senza impiegare alcuno sforzo per discernere se la norma richiesta sia stata seguita o meno, perchè la verifica è \textbf{evidente}.
			 \item\textbf{‘a’ a pedice dell’elemento}: Identifica un elemento della lista che per essere verificato in un caso contingente potrebbe richiedere maggiore sforzo in termini temporali, questo per via della natura ambigua della norma o dell’ elemento considerato oppure perché potenzialmente la verifica potrebbe richiedere la consultazione di un altro membro del gruppo di lavoro.
		\end{itemize}
		\subsubsection{Liste di controllo documentazione}
		\subsubsection{Liste di controllo codice}
			
