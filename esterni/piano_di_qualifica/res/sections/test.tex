\section{Test}
	Per la classificazione dei test si fa riferimento alle sezioni Verifica (3.5) e Validazione(3.6) delle \dext{Norme di Progetto v1.0.0}.
	
	\subsection{Tipologie di test:}
		I test saranno di quattro tipologie differenti:
		\begin{itemize}
			\item \textbf{TU}: Test Unità;
			\item \textbf{TR}: Test di Regressione; 
			\item \textbf{TI}: Test di Integrazione; 
			\item \textbf{TS}: Test di Sistema;
			\item \textbf{TA}: Test di Accettazione.
		\end{itemize}
		per la classificazione, con riferimento alla sezione 3.5.3.8 delle \dext{Norme di Progetto v1.0.0}, si utilizzerà il seguente formato:
		\begin{center}
			\textbf{<TipologiaTest>-<numeroProgressivo>}
		\end{center}
			
	\subsection{Test di Accettazione}
		I test di accettazione son utilizzati per dimostrare che tutti i requisiti individuati dal capitolato, concordati con il proponente, siano soddisfatti. \\
		Saranno infatti poi questi requisiti ad essere infatti testati in fase di collaudo del prodotto finale.
	\newpage
	\subsection{Test di Sistema}
		\begin{center}
		Riepilogo Test di Sistema
			\rowcolors{2}{white}{lightest-grayest}
			\begin{longtable}{|c|p{10cm}|c|}
			\hline
			\rowcolor{lighter-grayer}
			\textbf{Codice} & \textbf{Descrizione} & \textbf{Stato}  \\

			\hline
			\endhead

	
			\hline
			% registrazione
			TS-1 & Si verifichi che l'utente possa creare un \glock{account} inserendo:
			  \begin{itemize}
			 	\item\textbf{1.1} Nome utente
			 	\item\textbf{1.2}  \glock{Email}
			 	\item\textbf{1.3}  \glock{Password}
			 \end{itemize}& NI \\
			 \hline
			 TS-2 & Si verifichi che venga mostrato all'utente dopo la conferma della registrazione un messaggio d'errore se:
			  \begin{itemize}
			 	\item\textbf{2.1} il nome utente è già in uso
			 	\item\textbf{2.2}  l'\glock{email} è già in uso
			 	\item\textbf{2.3}  la \glock{password} non è valida
			 \end{itemize}& NI \\
			 \hline
			 
			 
			 
			 % login
			TS-3 & Si verifichi che l'utente possa fare il \glock{login} con il proprio \glock{account} inserendo
			 \begin{itemize}
			 	\item\textbf{3.1} Nome utente
			 	\item\textbf{3.2}  \glock{Password}
			 \end{itemize}& NI \\
			 \hline
			 TS-4 & Si verifichi che venga mostrato all'utente dopo la conferma della registrazione un messaggio d'errore se:
			 \begin{itemize}
			 	\item\textbf{4.1} il nome utente non esiste
			 	\item\textbf{4.2}  la \glock{password} non è corretta
			 \end{itemize}& NI \\
			 \hline
			 
			 % logout
			 TS-5 & Si verifichi che l'utente possa effettuare il \glock{logout}, dopo aver fatto il \glock{login} & NI \\
			 \hline
			 
			 % logout
			 TS-6 & Si verifichi che l'utente possa eliminare il proprio profilo, dopo aver fatto il \glock{login} & NI \\
			 \hline
			 
			 % impostazioni
			 TS-7 & Si verifichi che l'utente possa modificare le seguenti impostazioni: 
			  \begin{itemize}
			 	\item\textbf{7.1} volume musica
			 	\item\textbf{7.2} volume effetti sonori
			 	\item\textbf{7.3} visualizzazione avversari
			 \end{itemize}
			 & NI \\
			 \hline
			 
			 % classifica
			 TS-8 & Si verifichi che l'utente possa visualizzare le seguenti classifiche:
			  \begin{itemize}
			 	\item\textbf{8.1} punteggio per giocatore per ogni partita
			 	\item\textbf{8.2} partite più durature di tutti i giocatori
			 \end{itemize}
			 & NI \\
			 \hline
			 
			 % inizio partita
			 TS-9 & Si verifichi che agli utenti non autenticati venga associato un identificativo autogenerato & NI \\
			 \hline
			 TS-10 & Si verifichi che l'utente possa iniziare una partita:
			  \begin{itemize}
			 	\item\textbf{10.1} \glock{single player}
			 	\item\textbf{10.2}  \glock{multi player}
			 \end{itemize}
			 & NI \\
			 \hline
			 
			 % gestione partita
			 TS-11 & Si verifichi che un utente possa abbandonare una partita & NI \\
			 \hline
			 
			 % iterazione partita
			 TS-12 & Si verifichi che un utente durante una partita possa sposare il proprio giocatore & NI \\
			
			 TS-13 & Si verifichi che un utente in partita possa accumulare \glock{power-up} & NI \\
			 \hline
			 TS-14 & Si verifichi che un utente in partita perda vita in caso di collisione con nemico & NI \\
			 \hline
			 TS-15 & Si verifichi che un utente possa giocare al massimo contro 5 avversari in \glock{multi-player} & NI \\
			 \hline
			 TS-16 & Si verifichi che un utente possa giocare all'infinito & NI \\
			 \hline
			 TS-17 & Si verifichi che il sistema avvia la partita dopo 30 secondi a partire da quando c’è un numero minimo di giocatori nella lobby & NI \\
			 \hline 
			 TS-18 & Si verifichi che il \glock{power-up Cuore} deve aumentare di un’unità la vita del giocatore & NI \\
			 \hline
			 TS-19 &Si verifichi che il \glock{power-up Cuore} può comparire quando il giocatore ha perso almeno un’unità di vita. & NI \\
			  \hline
			 TS-20  & Si verifichi che il \glock{power-up Cambio arma casuale} modifichi casualmente la modalità di fuoco della navicella  &NI \\
			  \hline
			 TS-21  & Si verifichi che il \glock{power-up Scudo} azzera i danni subiti per un determinato lasso di tempo.  &NI \\
			 \hline
			 TS-22  & Si verifichi che la difficoltà cresca all'aumentare del punteggio della partita.  &NI \\
			 \hline
			 TS-23  & Si verifichi che se il giocatore è colpito da un nemico, gli venga sottratto un $\delta$ dalla vita.  &NI \\
			 \hline
			 TS-24 & Si verifichi che il giocatore possa giocare contro al massimo altri 5 giocatori & NI \\
			 \hline
			 TS-25 & Si verifichi che il periodo che intercorre tra la ricezione della richiesta e l'inizio dell'invio della risposta sia inferiore a  200 ms. & NI \\ 
			 \hline
			 TS-26 & Si verifichi che la documentazione delle \glock{API} sia esaustiva e completa & NI \\
			 \hline
			 TS-27 & Si verifichi che il codice sorgente venga gestito tramite un sistema di versionamento. & NI \\
			 \hline
			 TS-28 & Si verifichi che il codice sorgente sia testato prima di esser integrato. & NI \\
			 \hline
			 TS-29 & Si verifichi che siano stati realizzati dei test di unità e di integrazione per verificare le singole componenti e i \glock{subsystem} interni. & NI \\
			 \hline
			 TS-30 & Si verifichi che il sistema sia basato a \glock{microservizi} & NI \\
			 \hline
			 TS-31 & Si verifichi che l'infrastruttura abbia almeno: 
			 \begin{itemize}
			 	\item\textbf{31.1} almeno un \glock{microservizio} per l'autenticazione;
			 	\item\textbf{31.2} almeno un \glock{microservizio} per la gestione delle classifiche;
			 	\item\textbf{31.2} almeno un \glock{microservizio} per la gestione della partita;
			 \end{itemize}& NI \\
			 \hline
			 TS-32 & Si verifichi che il sistema supporti almeno 6 giocatori & NI \\
			 \hline
			 TS-33 & Si verifichi che il sistema permetta di distinguere un giocatore da un'altro & NI \\
			 \hline
			 TS-34 & Si verifichi che la scelta della tecnologia del servizio di hosting sia ben fondata e documentata & NI \\
			 \hline
			 TS-35 & Si verifichi che la scelta della tecnologia di sviluppo dell'applicazione sia fondata e documentata & NI \\
			 \hline
			 TS-36 & Si verifichi che la scelta della tecnologia di sviluppo lato server sia fondata e documentata & NI \\
			 \hline
			 TS-37 & Si verifichi che i \glock{microservizi} siano indipendenti tra di loro & NI \\ 
			 \hline
			 TS-38 & Si verifichi che sia fornito un bug reporting dello sviluppo & NI \\ 
			 \hline
			 \rowcolor{white}
			 \caption{Tabella contenente un riepilogo dei test di sistema}
			\end{longtable}
		\end{center}


	\subsection{Test di Integrazione}
		Le specifiche di questi test verranno definite successivamente rispettando il \glock{modello a V}. 

	\subsection{Test di Unità}
	 	Le specifiche di questi test verranno definite successivamente rispettando il \glock{modello a V}. 
