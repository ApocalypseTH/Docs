\section{Test}
	Per la classificazione dei test si fa riferimento alle sezioni Verifica (3.5) e Validazione(3.6) delle \dext{Norme di Progetto v1.0.0}.
	
	\subsection{Tipologie di test:}
		I test saranno di quattro tipologie differenti:
		\begin{itemize}
			\item \textbf{TU}: Test Unità;
			\item \textbf{TR}: Test di Regressione; 
			\item \textbf{TI}: Test di Integrazione; 
			\item \textbf{TS}: Test di Sistema;
			\item \textbf{TA}: Test di Accettazione.
		\end{itemize}
		per la classificazione, con riferimento alla sezione 3.5.3.8 delle \dext{Norme di Progetto v1.0.0}, si utilizzerà il seguente formato:
		\begin{center}
			\textbf{<TipologiaTest>-<numeroProgressivo>}
		\end{center}
			
	\subsection{Test di Accettazione}
		I test di accettazione son utilizzati per dimostrare che tutti i requisiti individuati dal capitolato, concordati con il proponente, siano soddisfatti. \\
		Saranno infatti poi questi requisiti ad essere infatti testati in fase di collaudo del prodotto finale.
	\newpage
	\subsection{Test di Sistema}
		\begin{center}
		Riepilogo Test di Sistema
			\rowcolors{2}{white}{lightest-grayest}
			\begin{longtable}{|c|p{10cm}|c|}
			\hline
			\rowcolor{lighter-grayer}
			\textbf{Codice} & \textbf{Descrizione} & \textbf{Stato}  \\

			\hline
			\endhead

	
			\hline
			TS-1 & Si verifichi che l'utente possa fare il \glock{login} con il proprio \glock{account} & NI \\
			 \hline
			 TS-2 & Si verifichi che l'utente possa creare un \glock{account} & NI \\
			 \hline
			 TS-3 & Si verifichi che l'utente possa modificare le impostazioni di gioco & NI \\
			 \hline
			 TS-4 & Si verifichi che l'utente autenticato possa accedere alla sezione per la modifica dei dati per proprio \glock{account}:
			 \begin{itemize}
			 	\item\textbf{4.1} Visualizzazione e cambio nome utente
			 	\item\textbf{4.2}  Visualizzazione e cambio \glock{email}
			 	\item\textbf{4.3}  Cambio \glock{password}
			 \end{itemize}
			 & NI \\
			 \hline
			 TS-5 & Si verifichi che un utente autenticato possa accedere ad una dashboard che mostra:
			 \begin{itemize}
			 	\item\textbf{5.1}  la possibilità di iniziare una partita \glock{single-player};
			 	\item\textbf{5.2}  la possibilità di iniziare una partita \glock{multi-player};
			 	\item\textbf{5.3}  la possibilità di silenziare il gioco;
			 \end{itemize} & NI \\
			 \hline
			 TS-6 & Si verifichi che un utente possa visualizzare la \glock{dashboard} dei risultati di gioco  & NI \\
			 \hline
			 TS-7 & Si verifichi che un utente possa una classifica generale di tutto il gioco  & NI \\
			 \hline
			 TS-8 & Si verifichi che un utente autenticato in partita singola possa mettere in pausa il gioco & NI \\
			 \hline
			 TS-9 & Si verifichi che un utente in partita possa spostare la sua posizione corrente & NI \\
			 \hline
			 TS-10 & Si verifichi che un utente in partita perda vita in caso di collisione con nemico & NI \\
			 \hline
			 TS-11 & Si verifichi che un utente in partita possa accumulare \glock{power-up} & NI \\
			 \hline
			 TS-12 & Si verifichi che un utente possa giocare al massimo contro 5 avversari in \glock{multi-player} & NI \\
			 \hline
			 TS-13 & Si verifichi che un utente possa giocare all'infinito & NI \\
			 \hline
			  %%%%%%%%%%%%%%%%%%%%%%%%%%%%%%%%%%%%%%% Test di sistema per i requisiti prestazionali %%%%%%%%%%%%%%%%%%%%%%%%%%%%%%%%%%%%%%%%%%%%%%%
			 TS-14 & Si verifichi che il server supporti carichi pesanti di richieste & NI \\ 
			 \hline
			 TS-15 & Si verifichi che il server supporti richieste da dispositivi di natura diversa & NI \\
			 \hline
			 TS-16 & Si verifichi che l'applicazione non sia eccessivamente \glock{CPU-intensive} comportando surriscaldamento & NI \\
			 \hline
			 TS-17 & Si verifichi che i tempi di risposta della web app sia inferiore ai 3 secondi & NI \\
			 \hline
			 TS-18 & Si verifichi che la documentazione delle \glock{API} sia esaustiva e completa & NI \\
			 \hline
			 TS-19 & Si verifichi che il codice sorgente venga gestito tramite un sistema di versionamento. & NI \\
			 \hline
			 TS-20 & Si verifichi che il codice sorgente sia testato prima di esser integrato. & NI \\
			 \hline
			 TS-21 & Si verifichi che siano stati realizzati dei test di unità e di integrazione per verificare le singole componenti e i \glock{subsystem} interni. & NI \\
			 \hline
			 TS-22 & Si verifichi che l'\glock{API} risponda tramite \glock{JSON} valido & NI \\
			 \hline
			 TS-23 & Si verifichi che l'\glock{API} sia sviluppata tramite \glock{NodeJS} & NI \\
			 \hline
			 TS-24 & Si verifichi che l'infrastruttura funzioni a \glock{microservizi}: 
			 \begin{itemize}
			 	\item\textbf{24.1} almeno un \glock{microservizio} per l'autenticazione;
			 	\item\textbf{24.2} almeno un \glock{microservizio} per la gestione delle classifiche;
			 \end{itemize}& NI \\
			 \hline
			 TS-25 & Si verifichi che i servizi di hosting siano basati su \glock{AWS} & NI \\ 
			 \hline
			 TS-26 & Si verifichi che il sistema faccia uso di \glock{HTTP} per la comunicazione con l'\glock{API}. & NI \\
			 \hline
			 TS-27 & Si verifichi che il sistema sia basato a \glock{microservizi} & NI \\
			 \hline
			 TS-28 & Si verifichi che il sistema faccia uso \glock{API} per la comunicazione con i \glock{client} & NI \\
			 \hline
			 TS-29 & Si verifichi che le \glock{API} rispondano usando standard \glock{JSON} & NI \\
			 \hline
			 TS-30 & Si verifichi che il sistema supporti almeno 6 giocatori & NI \\
			 \hline
			 TS-31 & Si verifichi che il sistema permetta la comunicazione tra client con Sistemi Operativi di versioni diverse & NI \\
			 \hline
			 TS-32 & Si verifichi che il sistema permetta l'invio di dati dal client& NI \\
			 \hline
			 TS-33 & Si verifichi che il sistema permetta di distinguere un giocatore da un'altro & NI \\
			 \hline
			 TS-34 & Si verifichi che ogni parte del sistema sia testata adeguatamente & NI \\
			 \hline
			 TS-35 & Si verifichi che la scelta della tecnologia di hosting sia ben fondata e documentata & NI \\
			 \hline
			 TS-36 & Si verifichi che i \glock{microservizi} siano indipendenti tra di loro & NI \\ 
			 \hline
			 TS-37 & Si verifichi che la connessione al \glock{server} avvenga tramite l'applicazione \glock{client} & NI \\
			 \hline
			 \caption{Tabella contenente un riepilogo dei test di sistema}
			\end{longtable}
		\end{center}


	\subsection{Test di Integrazione}
		Le specifiche di questi test verranno definite successivamente rispettando il \glock{modello a V}. 

	\subsection{Test di Unità}
	 	Le specifiche di questi test verranno definite successivamente rispettando il \glock{modello a V}. 
