\section{Introduzione}
	\subsection{Scopo del documento}
		Il documento mira a presentare i metodi di verifica e validazione adottati dal gruppo EverBuilds per garantire la qualità di prodotto e di processo. 
		Per esser sicuri di mantenere i vincoli, verrà quindi predisposta una validazione e verifica da parte di un verificatore prima della convalida di qualsiasi lavoro svolto.
	\subsection{Scopo del prodotto}
		Lo scopo del prodotto è lo sviluppo di un videogioco mobile che dovrà offrire un esperienza a singolo giocatore, e multi-giocatore sincrona, che dopo essersi autenticati, potranno giocare nel seguente modo:
		\begin{itemize}
			\item \textbf{singolo}:
				il giocatore proseguirà nel gioco finché non avrà finito le vite.
			\item \textbf{multi-giocatore}:
				i vari giocatori si scontreranno contro i medesimi avversari, fino a ché ne sia rimasto solo uno, il quale sarà il vincitore.
		\end{itemize}
		
		
		
	\subsection{Glossario e Documenti esterni}
		Alcuni termini presenti nei documenti, potrebbero essere ambigui: per evitare tale situazione, dopo tali termini, verrà segnata una \textit{G} al pedice, che significherà che il significato atteso, e descritto all'interno del \dext{Glossario v1.0.0}.
		Se invece di fianco a un termine è presente una  \textit{D}, significa che allora tale termine si riferisce a un documento.
		
	\subsection{Riferimenti}
		\subsubsection{Normativi}
		\begin{itemize}
			\item \textbf{Norme di progetto:} \dext{Norme di Progetto v1.0.0};
			\item \textbf{Capitolato d’appalto C6 - RGP - Realtime Gaming Platform:}\\
			\href{https://sesaspa-my.sharepoint.com/personal/s_dindo_vargroup_it/_layouts/15/onedrive.aspx?id=%2Fpersonal%2Fs%5Fdindo%5Fvargroup%5Fit%2FDocuments%2FDownload%2Fupload%2FIngegneria%5Fsoftware%2FCapitolato%5FIngegneria%5Fsoftware%2Epdf&parent=%2Fpersonal%2Fs%5Fdindo%5Fvargroup%5Fit%2FDocuments%2FDownload%2Fupload%2FIngegneria%5Fsoftware&originalPath=aHR0cHM6Ly9zZXNhc3BhLW15LnNoYXJlcG9pbnQuY29tLzpiOi9nL3BlcnNvbmFsL3NfZGluZG9fdmFyZ3JvdXBfaXQvRVRodmF5MGY2S1ZDb1h5ZFlPY2UybGtCdC1NWWNuVzF5YWZSWEZYVklPSXNIZz9ydGltZT1jQjNzY3NxczJFZw}{https://sesaspa-my.sharepoint.com/}
		\end{itemize}
		\subsubsection{Informativi}
		\begin{itemize}
			\item \textbf{ISO/IEC 9126:}\\
			\url{https://en.wikipedia.org/wiki/ISO/IEC_9126}
			
			\item ISO/IEC 12207:1995 e sue evoluzioni \\
			\url{https://www.math.unipd.it/~tullio/IS-1/2009/Approfondimenti/ISO_12207-1995.pdf}
			
			\item \textbf{Slide del corso di Ingegneria del Software, qualità del software:}\\
			\url{https://www.math.unipd.it/~tullio/IS-1/2020/Dispense/L12.pdf}
			
			\item \textbf{Slide del corso di Ingegneria del Software, qualità di processo:}\\
			\url{https://www.math.unipd.it/~tullio/IS-1/2020/Dispense/L13.pdf}
		
		\end{itemize}
	