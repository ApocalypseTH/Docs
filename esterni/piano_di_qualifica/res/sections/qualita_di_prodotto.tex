\section{Qualità di prodotto}
	Per garantire e valutare la qualità del prodotto il gruppo ha selezionato lo standard ISO/IEC 9126, che definisce la qualità del prodotto tramite una lista di attributi
	Oltre a ciò, il gruppo ha deciso di assumere altri parametri per la valutazione della qualità della documentazione riguardante al prodotto finale.
	Di seguito la lista delle qualità che si è deciso di prendere in considerazione riguardo allo stato corrente del progetto.	
	\subsection{Qualità dei documenti}
	\subsubsection{QPD-1 Comprensione}
		Tutti i documenti devono essere leggibili e comprensibili da chiunque, per tale motivo si decide di adottare metriche per la valutazione della correttezza lessicografica, grammaticale, e semantica, in quanto da esse deriva la facilità di lettura.
		
	\paragraph{Obiettivi}
		\begin{itemize}
			\item \textbf{correttezza:} i documenti che saranno presentati non dovranno presentare errori ortografici di alcun genere
		\end{itemize}
		
		
	\paragraph{Metriche}
	La comprensione dei documenti viene valutata dai seguenti criteri:
	\begin{itemize}
		\item QM-PD-DOC01 \glock{Indice di Gulpease};
     		\item QM-PD-DOC02 Correttezza ortografica.
	\end{itemize}
	\paragraph{Indici di qualità}
	\begin{center}
		\rowcolors{2}{white}{white}
		\begin{longtable}{|c|c|c|}
			\hline
			\rowcolor{lighter-grayer}
			\textbf{Nome} & \textbf{Valore Preferibile} & \textbf{Valore Accettabile}\\
			\hline
			\endfirsthead
			\hline
			\glock{Indice di Gulpease} &  \(\ge 70\) & \(\ge 50\) \\
 		 	\hline
			\rowcolor{lightest-grayest}
			Correttezza ortografica & 0 & 0 \\
			\hline
			\caption{Indici di qualità per le metriche per la Gestione delle Risorse}
		\end{longtable}		
	\end{center}
	
	\subsection{Qualità del prodotto finale}
	\subsubsection{QPD-2 Completezza}
	Il prodotto finale deve essere il più possibile avvicinarsi a quello desiderato, in quanto a efficienza, aspettative, utilizzo, vincoli adempiti se il prodotto finale è conforme a quanto descritto nell'\dext{Analisi dei requisiti}: per questo motivo si procederà a definire delle metriche che si baseranno su quanto descritto in tale documento.
	
		\paragraph{Obiettivi}
			\begin{itemize}
				\item{Completezza dell'implementazione:}
					Il prodotto finale deve adempiere a tutti i vincoli e fornire tutte le caratteristiche prefissati, ed esposti dentro al \dext{Analisi dei Requisiti}
				\item{Accuratezza}: capacità del prodotto software di fornire i risultati desiderati con la precisione richiesta
			\end{itemize}
		
		\paragraph{Metriche}
			La valutazione della conformità del prodotto finale verrà valutata tramite le seguenti metriche
			\begin{itemize}
     				\item  QM-PD-SVL01  Percentuale requisiti soddisfatti
			\end{itemize}
		\paragraph{Indici di qualità}
			\begin{center}
				\rowcolors{2}{white}{lightest-grayest}
				\begin{longtable}{|c|c|c|}
				\hline
				\rowcolor{lighter-grayer}
				\textbf{Nome} & \textbf{Valore Ottimale} & \textbf{Valore Accettabile}\\
				\hline
				\endfirsthead
				\hline
				N\textsubscript{FI} & // & // \\
				\hline
				N\textsubscript{FTOT} & // & // \\
				\hline				\hline
				C & 100\% & 100\% \\
				\hline
				\caption{Indici di qualità per completezza del prodotto finale}
				\end{longtable}
			\end{center}
				
	
	
	
	
	
	
	
	
	
	
	
	
	
	
	
	