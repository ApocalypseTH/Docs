
\section{Resoconto attività di verifica}

\subsection{Verifica documentazione}
 Come descritto nelle \dext{Norme di Progetto v1.0.0}, vengono usate la tecnica del  \glock{walkthrough} per la validazione dei vari documenti con cadenza periodica, per una verifica generale dell'adempimento a tutte le regole, e della tecnica dell' \glock{inspection} per la verifica delle varie modifiche/aggiunte ai documenti
\newline
\newline
\emph{Alcuni documenti non erano disponibili fin dall'inizio, e quindi i relativi valori delle metriche non son riportati}


\subsubsection{Calcolo leggibilità documenti}
Come descritto nelle \dext{Norme di Progetto v1.0.0}, si utilizza \glock{l'indice di Gulpease} per la verifica che i documenti siano leggibili e comprensibili da tutti. 
Di seguito viene riportato il grafico contenente i risultati ottenuti durante il periodo di analisi dei requisiti nei 6 periodi definiti:

\begin{figure}[H]
	\centering
	\includegraphics[width=0.8\linewidth]{./res/images/gulpease.png}
	\caption{Grafico periodo/\glock{indice di Gulpease} nel periodo di analisi dei requisiti}
	\label{fig:Grafico indice di Gulpease periodo di analisi dei requisiti}
\end{figure}

\subsubsection{Calcolo ortografia documenti}
Di seguito riportati risultati dei controlli periodici dei vari documenti riguardo a errori ortografici nel periodo di analisi dei requisiti:
\begin{figure}[H]
	\centering
	\includegraphics[width=0.8\linewidth]{./res/images/ortografia.png}
	\caption{Grafico periodo/errori ortografici nel periodo di analisi dei requisiti}
	\label{fig:Grafico errori ortografici periodo di analisi dei requisiti}
\end{figure}

Condizione documenti finali:
\begin{center}
	\rowcolors{2}{lightest-grayest}{white}
	\begin{longtable}{|c|c|c|}
	\hline
	\rowcolor{lighter-grayer}
	\textbf{Documento} & \textbf{Errori ortografici} & \textbf{Accettabile} \\
	\hline
	\endfirsthead

	\hline
	\dext{Piano di Progetto v1.0.0} & 0 & Sì \\
	\hline
	\hline
	\dext{Norme di Progetto v1.0.0} &  0 & Sì \\
	\hline
	\hline
	\dext{Studio di fattibilità v1.0.0} & 0 & Sì \\
	\hline
	\hline
	\dext{Glossario v1.0.0} & 0 & Sì \\
	\hline
	\hline
	\dext{Piano di Qualifica v1.0.0} & 0 & Sì \\
	\hline
	\hline
	\dext{Verbali} & 0 & Sì \\
	\hline
	\hline
	\dext{Analisi dei requisiti v1.0.0} & 0 & Sì \\
	\hline
	\rowcolor{white}
	\caption{Tabella degli errori di ortografia alla fine del periodo di analisi dei requisiti}
	\end{longtable}
\end{center}








%-- prodotto finale



\subsection{Verifica del Prodotto finale}	
	\subsubsection{Calcolo requisiti soddisfatti nel tempo}
		Periodicamente, son stati raccolti dati dell'avanzamento del progetto, con un focus ai requisiti che son stati soddisfatti fino a quel momento, rispetto a quelli che ci si era preventivati di finire nel \dext{Piano di Progetto v1.0.0}.
		Di seguito vengono mostrati i risultati di tali verifiche:
		\begin{figure}[H]
			\centering
			\includegraphics[width=0.8\linewidth]{./res/images/rtot.png}
			\caption{Grafico che mostra quanti requisiti son stati soddisfatti in relazione a quelli definiti nel \dext{Piano di Progetto v1.0.0} nel periodo di Analisi dei requisiti}
			\label{fig:Grafico requisiti soddisfatti nel periodo di Analisi dei Requisiti}
		\end{figure}

		\subsubsection{Calcolo ortografia documenti}
		Di seguito riportati risultati dei controlli periodici dei vari documenti riguardo a errori ortografici nel periodo di analisi dei requisiti:
		\begin{figure}[H]
			\centering
			\includegraphics[width=0.8\linewidth]{./res/images/prs.png}
			\caption{Grafico periodo/percentuale requisiti soddisfatti nel periodo di analisi dei requisiti}
			\label{fig:Grafico PRS periodo di analisi dei requisiti}
		\end{figure}

		Condizione prodotto finali:
		\begin{center}
			\rowcolors{2}{lightest-grayest}{white}
			\begin{longtable}{|c|c|c|}
			\hline
			\rowcolor{lighter-grayer}
			\textbf{Indice} & \textbf{PRS} & \textbf{Accettabile} \\
			\hline
			\endfirsthead
			\hline
			PRS & 100\% & Si \\
			\hline
			\rowcolor{white}
			\caption{Tabella del completamento dei requisiti alla fine del periodo di analisi dei requisiti}
		\end{longtable}
	\end{center}
