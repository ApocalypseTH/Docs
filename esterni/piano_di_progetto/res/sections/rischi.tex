\section{Analisi dei rischi}

	Lo sviluppo di progetti complessi include una possibile moltitudine di problemi dalla progettazione, alla programmazione addirittura come risultato di attriti tra rapporti interpersonali e opinioni. Grazie ad una attenta analisi delle principali fonti di rischio verranno gestiti dal gruppo come segue:
	\begin{itemize}
		\item \textbf{Identificazione:} individuare i potenziali rischi e fattori problematici che possono presentarsi durante l’avanzamento del progetto.
		\item \textbf{Analisi:} attività di studio e valutazione delle probabilità di occorrenza e delle possibili conseguenze sul progetto.
		\item \textbf{Pianificazione:} istituire metodologie di controllo per evitare il verificarsi dei rischi individuati.
		\item \textbf{Controllo:} monitoraggio continuo al fine di evitare il presentarsi di queste complicazioni o, nel caso peggiore, agire tempestivamente per la risoluzione.
	\end{itemize}
	
	Inoltre sono stati definiti i seguenti codici identificativi per raggruppare ed individuare in modo più rapido i singoli rischi:
	\begin{itemize}
		\item \textbf{RI}: Rischi Interpersonali;
		\item \textbf{RO}: Rischi Organizzativi;
		\item \textbf{RT}: Rischi Tecnologici.
	\end{itemize}


\begin{center}
	\begin{longtable}{| p{5cm} | p{5cm} | p{5cm} |}
		\hline
		\rowcolor[HTML]{F9CB9C} 
		Tipologia & Descrizione & Rilevazione \\ \hline
		Inesperienza a livello tecnico & Molte delle tecnologie adottate e conoscenze richieste dal progetto non sono familiari per molti membri del gruppo. & Per evitare limitazioni in stile “collo di bottiglia” dove solo chi conosce le tecnologie potrà contribuire alla produzione del software ogni componente del gruppo si impegna a comunicare ogni eventuale incomprensione e difficoltà. \\ 
		\hline
		\rowcolor[HTML]{F9CB9C} 
		Gravità & Probabilità & ID  \\ \hline
		Alta & Alta & RT1  \\ \hline
		\cellcolor[HTML]{F9CB9C} Piano di Contingenza & \multicolumn{2}{|p{10cm}|}{Ogni componente del gruppo provvederà a sanare le proprie lacune tramite un lavoro di autoformazione entro un dato livello concordato.} \\ \hline
		\caption{Trattamento del rischio: inesperienza a livello tecnico.}
		\label{fig: Trattamento del rischio: inesperienza a livello tecnico.}
	\end{longtable}
\end{center}

\begin{center}
	\begin{longtable}{| p{5cm} | p{5cm} | p{5cm} |}
		\hline
		\rowcolor[HTML]{F9CB9C} 
		Tipologia & Descrizione & Rilevazione \\ \hline
		Impreparazione a livello gestionale & Non avendo mai affrontato progetti di questo genere Il gruppo non ha le conoscenze adeguate per una gestione efficiente del lavoro. & Sarà compito del responsabile, aiutato gruppo, cercare di suddividere in modo efficiente e adeguato il lavoro da svolgere tra i vari membri. \\ \hline
		\rowcolor[HTML]{F9CB9C} 
		Gravità & Probabilità & ID  \\ \hline
		Alta & Alta & RO1  \\ \hline
		\cellcolor[HTML]{F9CB9C} Piano di Contingenza & \multicolumn{2}{|p{10cm}|}{Ciascun membro del gruppo si occuperà di rispettare le scadenze e i lavori assegnati e in caso di difficoltà non esiterà a comunicare incomprensioni.} \\ \hline
		\caption{Trattamento del rischio: impreparazione a livello gestionale.}
		\label{fig: Trattamento del rischio: impreparazione a livello gestionale.}
	\end{longtable}
\end{center}

\begin{center}
	\begin{longtable}{| p{5cm} | p{5cm} | p{5cm} |}
		\hline
		\rowcolor[HTML]{F9CB9C} 
		Tipologia & Descrizione & Rilevazione \\ \hline
		Impegni Accademici e personali & La disponibilità dei membri del gruppo potrebbe variare per impegni che hanno nella propria vita privata o con  l'università, per esempio: esami arretrati, studio per un appello o progetti di altri corsi. & É stato organizzato un calendario ove notificare eventuali impegni personali e accademici in anticipo. \\ \hline
		\rowcolor[HTML]{F9CB9C} 
		Gravità & Probabilità & ID  \\ \hline
		Bassa & Alta & RI1  \\ \hline
		\cellcolor[HTML]{F9CB9C} Piano di Contingenza & \multicolumn{2}{|p{10cm}|}{L’assegnazione degli incarichi avverrà nel rispetto degli impegni presenti sul calendario. } \\ \hline
		\caption{Trattamento del rischio: impegni Accademici e personali.}
		\label{fig: Trattamento del rischio: impegni Accademici e personali.}
	\end{longtable}
\end{center}

\break

\begin{center}
	\begin{longtable}{| p{5cm} | p{5cm} | p{5cm} |}
		\hline
		\rowcolor[HTML]{F9CB9C} 
		Tipologia & Descrizione & Rilevazione \\ \hline
		Calcolo costi e monte ore & Considerando che il gruppo non ha mai trattato in precedenza valutazioni di tipo economico il calcolo dei costi e il monte ore complessivo potrebbero essere imprecise.  & Il gruppo ha predisposto tabelle che ogni membro deve compilare per permette al responsabile di monitorare le ore di lavoro svolte. \\ \hline
		\rowcolor[HTML]{F9CB9C} 
		Gravità & Probabilità & ID  \\ \hline
		Alta & Alta & RO2  \\ \hline
		\cellcolor[HTML]{F9CB9C} Piano di Contingenza & \multicolumn{2}{|p{10cm}|}{All’insorgere di variazioni significative nel monte ore totale il responsabile lo comunicherà tempestivamente al committente. } \\ \hline
		\caption{Trattamento del rischio: calcolo costi e monte ore.}
		\label{fig: Trattamento del rischio: calcolo costi e monte ore.}
	\end{longtable}
\end{center}

\begin{center}
	\begin{longtable}{| p{5cm} | p{5cm} | p{5cm} |}
		\hline
		\rowcolor[HTML]{F9CB9C} 
		Tipologia & Descrizione & Rilevazione \\ \hline
		Comunicazione e contrasti interni & Essendo la prima occasione di lavoro per il gruppo è possibile che sorgano contrasti, incomprensioni o che alcuni membri siano meno disposti alla comunicazione.  & Tutti i componenti del gruppo sono tenuti ad impegnarsi nel rispondere il più velocemente possibile alle comunicazioni interne e ad evitare che i possibili contrasti interni non influiscono sullo svolgersi delle attività. \\ \hline
		\rowcolor[HTML]{F9CB9C} 
		Gravità & Probabilità & ID  \\ \hline
		Alta & Alta & RI2  \\ \hline
		\cellcolor[HTML]{F9CB9C} Piano di Contingenza & \multicolumn{2}{|p{10cm}|}{Il responsabile avrà la funzione di mediatore e in caso di mancate risposte dovrà occuparsi di contattare personalmente il membro in questione. } \\ \hline
		\caption{Trattamento del rischio: comunicazione e contrasti interni.}
		\label{fig: Trattamento del rischio: comunicazione e contrasti interni.}
	\end{longtable}
\end{center}

\break

\begin{center}
	\begin{longtable}{| p{5cm} | p{5cm} | p{5cm} |}
		\hline
		\rowcolor[HTML]{F9CB9C} 
		Tipologia & Descrizione & Rilevazione \\ \hline
		Ritardi nel corso di sviluppo & Molte delle problematiche trattate in precedenza possono sfociare in un ritardo generale sulla tabella di marcia. & Ogni membro dovrà comunicare tempestivamente l'incapacità di completare l'attività ad esso assegnata entro il tempo prestabilito. \\ 
		\hline
		\rowcolor[HTML]{F9CB9C} 
		Gravità & Probabilità & ID  \\ \hline
		Alta & Alta & RO3  \\ \hline
		\cellcolor[HTML]{F9CB9C} Piano di Contingenza & \multicolumn{2}{|p{10cm}|}{ Il responsabile si impegnerà a ridistribuire attività e risorse al fine di evitare rallentamenti. } \\ \hline
		\caption{Trattamento del rischio: ritardi nel corso di sviluppo.}
		\label{fig: Trattamento del rischio: ritardi nel corso di sviluppo.}
	\end{longtable}
\end{center}

\newpage