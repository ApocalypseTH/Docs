\section{Consuntivo}

\subsection{Fase di individuazione degli strumenti e fase di analisi}
Verranno riportate di seguito le spese effettivamente sostenute sia per ruolo che per persona. Il bilancio potrà risultare:
\begin{itemize}
	\item \textbf{positivo} se il preventivo supera il consuntivo;
	\item \textbf{negativo} se il consuntivo supera il preventivo;
	\item \textbf{in pari} se consuntivo e preventivo si equivalgono.
\end{itemize}

\subsubsection{Consuntivo a finire}
Verranno indicate le ore effettivamente sostenute e saranno da considerarsi come ore di approfondimento personale. Non saranno quindi rendicontate:

\begin{center}
	\begin{longtable}{| c | c | c |}
		\hline
		\rowcolor[HTML]{F9CB9C} 
		\textbf{Ruolo} & \textbf{Ore} & \textbf{Costo}  \\ \hline
		\textbf{Analista} & 94 (+5) & 2350€ (+125€)  \\ \hline
		\textbf{Amministratore} & 37 (+2) & 740€ (+40€)  \\ \hline
		\textbf{Programmatore} & - & -  \\ \hline
		\textbf{Progettista}  & 11 (+0) & 242€ (+0€)  \\ \hline
		 \textbf{Responsabile}  & 16 (+0) & 480€ (+0€)  \\ \hline
		   \textbf{Verificatore} & 94 (+3) & 1410€ (+45€)  \\ \hline
		     \textbf{Totale preventivo}  & \textbf{252} &  \textbf{5222€} \\ \hline
		     \textbf{Totale consuntivo}  & \textbf{262} &  \textbf{5430€} \\ \hline
		\rowcolor[HTML]{F9CB9C}
		\textbf{Differenza} & \textbf{10}  &  \textbf{210€} \\ \hline
		\caption{Consuntivo a finire della fase di individuazione degli strumenti e fase di analisi.}
		\label{fig: Consuntivo a finire della fase di individuazione degli strumenti e fase di analisi.}
	\end{longtable}
\end{center}

\break

\subsubsection{Conclusioni}
Come mostrato dalla tabella soprastante, nella prima fase di lavoro è stato necessario investire più tempo del previsto nei ruoli di amministratore, analista e verificatore.\break
Il bilancio risultante è negativo e di seguito sono spiegati i motivi di tale ritardo:
\begin{itemize}
	\item \textbf{Amministratore}: è stato necessario modificare alcune sezioni in \glock{Norme di Progetto} v1.0.0 per chiarire alcune problematiche riscontrate durante la stesura dei documenti;
	\item \textbf{Analista}: l’individuazione dei requisiti si è rivelata più difficile del previsto, quindi sono state necessarie più ore di lavoro;
	\item \textbf{Verificatore}: l’aggiunta di sezioni in \glock{Norme di Progetto} v1.0.0 il numero di requisiti trovati hanno richiesto più ore di lavoro da parte dei Verificatori.
\end{itemize}
	\newpage