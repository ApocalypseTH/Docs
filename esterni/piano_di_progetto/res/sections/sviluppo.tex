\section{Modello di sviluppo}

	Il modello di ciclo di vita scelto per il progetto è il modello incrementale, che favorisce il versionamento del prodotto scandendo lo sviluppo in \glock{fasi} suddivise in \glock{attività} e \glock{sottoattività} al cui termine saranno segnate da una \glock{milestone}.
	Questa decisione è stata presa nel rispetto del \glock{proponente} il quale potrà dare un riscontro al termine di ogni fase, inoltre la sua consultazione e approvazione saranno necessarie al fine di aggiungere, modificare o eliminare i requisiti; questo tuttavia non sarà possibile durante la fase di sviluppo dell’incremento corrente. 
	\\
	\\
	Obiettivi e vantaggi del modello incrementale:
	\begin{itemize}
		\item priorità nello sviluppo di funzionalità primarie, in modo da potersi basare subito sul feedback del proponente;
		\item successivo riscontro con il \glock{proponente} al termine di ogni incremento corrispondente al raggiungimento di una \glock{milestone};
		\item la propagazione degli errori è limitata al singolo incremento, dato che al termine di ognuno ne seguirà un’attenta verifica;
		\item ogni incremento fornisce indicazioni sullo sviluppo del successivo tramite l’\glock{approccio adattivo}.
	\end{itemize}

\newpage