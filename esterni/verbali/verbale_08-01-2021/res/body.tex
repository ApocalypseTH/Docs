\section{Introduzione}

\subsection{Luogo e data dell'incontro}
	\begin{itemize}
		\item \textbf{luogo:} Google Meet
		\item \textbf{data:} 2021-01-08
		\item \textbf{ora di inizio:} 17:00
		\item \textbf{ora di fine:} 18:20
	\end{itemize}

\subsection{Segretario}
Il segretario del verbale è \textbf{Alberto Sinigaglia}

\subsection{Presenze}
	Gruppo EverBuilds:
	\begin{itemize}
		\item \textbf{totale presenti:} 7
		\item \textbf{presenti: }
			\begin{itemize}		
				\item Riccardo Calcagno
				\item Giovanni Michieletto
				\item Samuele Sartor
				\item Vittorio Schiavon
				\item Alberto Sinigaglia
				\item Marco Tesser
				\item Alice Zago
			\end{itemize}
		\item \textbf{assenti: } 0
	\end{itemize}
	Zero12:
	\begin{itemize}
		\item \textbf{totale presenti:} 2
		\item \textbf{presenti: }
			\begin{itemize}		
				\item Michele Massaro
				\item Stefano Dindo
			\end{itemize}
	\end{itemize}


\newpage
\section{Resoconto}
	Nel meeting son stati trattati i seguenti argomenti:
	\begin{itemize}
		\item\textbf{Decisione metodo di comunicazione}:
			Si è deciso che da ora in poi per le comunicazione si utilizzerà Slack, per evitare problemi causati dall'utilizzo di email (come già avvenuto).\\
			L'azienda Zero12 si è presa l'onere della creazione del canale Slack.
		\item\textbf{Discussione sui focus principali del progetto}:
			La presentazione del capitolato è stata considerata un po vaga/poco definita da parte del gruppo EverBuilds, così si è proceduto al chiarimento degli obbiettivi realmente interessati del progetto. \\
			Il risultato è stato un netto interesse al lato tecnologico a lato server e cloud, e quindi il perchè abbiamo preso determinate decisioni, pro e contro delle tecnologie e tutto quello che è stato usato e che ha fondato le decisioni prese.
		\item\textbf{Discussione sulle tecnologie adottabili}
			Durante la presentazione, son state proposte solo alcune tecnologie utilizzabili per il compimento del progetto. \\
			In quanto opzionali, si è proceduto a discutere con i presenti di quali tecnologie fossero preferibili, con relativi vantaggi e svantaggi, in particolare:
			\begin{itemize}
				\item\textbf{Lato client}: \\
					discussione se procedere con sviluppo Android o IOS
				\item\textbf{Lato server}: \\
					discussione quali linguaggi era possibile utilizzare, da cui ne è uscita una preferenza per l'azienda Zero12 per NodeJS e Python
				\item\textbf{Lato DataBase}: \\
					discussione sulla preferenza tra database SQL, NoSQL, e pro e contro di entrambe le tecnologie
			\end{itemize}
		\item\textbf{Discussione sui microservizi}
			Durante la presentazione è stato dichiarato che l'infrastruttura dovesse essere basata su microservizi. \\
			In quanto "nuovi" all'ambito, il gruppo EverBuilds ha voluto chiarire quali microservizi ci si aspettasse; l'azienda ha quindi proceduto a chiarire che si aspettano almeno un microservizio per ognuno dei seguenti ambiti:
			\begin{itemize}
				\item autenticazione (è stata espressa anche una possibile alternativa, appoggiandosi a servizi esterni)
				\item gestione delle classifiche
				\item gestione delle partite 
			\end{itemize}
	\end{itemize}
	In quanto primo incontro, l'obbiettivo principale era quello di presentarsi e conoscersi, più che definire requisiti, o comunque "trarre decisioni".



