\section{P}
	\subsection{Password}
		 Una sequenza di caratteri alfanumerici utilizzata per accedere in modo esclusivo a una risorsa informatica o per effettuare operazioni di cifratura.
	\subsection{PDCA} 
		 Acronimo di Plan–Do–Check–Act è un metodo di gestione iterativo in quattro fasi utilizzato per il controllo e il miglioramento continuo dei processi e dei prodotti.
	\subsection{Pull-Request (PR)}  
		Richiesta di salvataggio delle modifiche in un branch del Version Control System.
	\subsection{Processi}  
		Insieme di attività che interagiscono per produrre un risultato.
	\subsection{Progettista} 
		Figura del gruppo il cui compito  è sviluppare una soluzione che soddisfi i bisogni individuati. \\
		Questa figura è responsabile delle attività di progettazione infatti il suo scopo è quello di produrre un’architettura che modelli il problema a partire da un insieme di requisiti.
	\subsection{Programmatore} 
		Figura incaricata delle attività di codifica e delle componenti necessarie per effettuare le prove di verifica. Egli deve implementare l’architettura prodotta dal Progettista in maniera che aderisca alle specifiche, ed è responsabile della manutenzione del codice creato.
	\subsection{Proof-of-concept (PoC) }  
		Con il termine proof-of-concept si intende una realizzazione incompleta o abbozzata di un determinato progetto o metodo, allo scopo di provarne la fattibilità o dimostrare la fondatezza di alcuni principi o concetti costituenti.





\newpage