\section{S}
	\subsection{Server}
		Componente o sottosistema informatico di elaborazione e gestione del traffico di informazioni che fornisce un qualunque tipo di servizio ad altre componenti.
	\subsection{Serverless} 
		Con il termine serverless si intende un network la cui gestione non viene incentrata su dei server, ma viene dislocata fra i vari utenti che utilizzano il network stesso, quindi il lavoro necessario di gestione del network viene eseguito dagli stessi utilizzatori.
	\subsection{Singleplayer}
		Termine usato per indicare la modalità di gioco in cui una sola persona prende parte al gioco per tutta la durata della partita.
	\subsection{SPICE}
		Acronimo di "Software Process Improvement and Capability Determination", è un insieme di documenti standard tecnici per il processo di sviluppo del software del computer e le relative funzioni di gestione aziendale.
	\subsection{Slack}
		Slack è una piattaforma di messaggistica per team che integra insieme diversi canali di comunicazione, che trattano argomenti differenti, in un unico servizio. 	
	\subsection{Stakeholder}  
		Con il termine stakeholder si intendono le parti interessate nella realizzazione di un progetto o più in generale di un endeavor (o impresa).
	\subsection{Subsystem}
		Termine inglese per rappresentare un sottosistema di un infrastruttura informatica.
	\subsection{Swift}
	    Linguaggio di programmazione creato da Apple.

\newpage