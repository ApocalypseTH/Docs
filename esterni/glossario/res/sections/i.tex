\section{I}
	\subsection{Incremento e Verifica}
		Redazione ulteriore dei vari documenti qualvolta sorga la necessità durante lo svolgimento del progetto, con annessa presa visione degli altri membri e conferma.
	\subsection{Indice di Gulpease}
		L'Indice Gulpease è un indice di leggibilità di un testo tarato sulla lingua italiana.
	\subsection{Individuazione degli strumenti}
		Fase preliminare all'inizio del vero progetto, comprende la stesura di tutta la documentazione di base e interessa un principio di autoformazione riguardo a tutte le tecnologie relativamente nuove per i membri, così da raggiungere la fase di progettazione con le idee sufficientemente chiare.
	\subsection{Inspection} 
		Tecnica di analisi statica che consiste nell'analizzare il prodotto nelle sue sole parti modificate, così da minimizzare e focalizzare la ricerca di errori nelle modifiche, assumendo corretto il restante.
	\subsection{IOS}
	    Sistema operativo per dispositivi mobili sviluppato da Apple.
	\subsection{Issue}  
		Con il termine issue si intende una qualsiasi richiesta o azione da svolgere, legata generalmente al concetto di issue in un ITS (Issue Tracking System).
	\subsection{ITS, Issue Tracking System}  
		Un Issue Tracking System o ITS, indica un sistema informatico il cui scopo è tracciare richieste di assistenza o problemi, ma non solo.
	
\newpage