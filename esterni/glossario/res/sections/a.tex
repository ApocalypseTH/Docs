\section{A}
	\subsection{Account}
		Termine rappresentate il complesso dei dati identificativi di un utente, che gli consentono l'accesso a un servizio telematico
	\subsection{Amministratore}  
		L'amministratore è la figura che garantisce l’affidabilità, l’efficienza e l’efficacia dei mezzi scelti dal gruppo e del rendimento dell’ambiente di lavoro.
	\subsection{Analista}  
		Figura di un gruppo il cui compito è quello di comprendere appieno la natura e complessità del problema evidenziandone i punti chiave.
    \subsection{Android}
        Android è un sistema operativo per dispositivi mobili sviluppato da Google.
	\subsection{API}  
		Acronimo di "application programming interface", in informatica, viene utilizzato per definire delle regole di comunicazione tra due entità, spesso client-server.
	\subsection{API Restful}  
		Delle RESTful API sono un set di API che rispettano il modello architetturale REST.
    \subsection{Arma}
        Elemento di gioco che rappresenta la tipologia di attacco di una navicella. Può essere di molte tipologie con varie specifiche.
	\subsection{Attore}
		Entità che interagisce con il sistema per svolgere delle attività.
	\subsection{Avversario}
	    Un utente che sta partecipando ad una partita multigiocatore.
	    
	\subsection{AWS}
		Acronimo di "Amazon Web Services", è un'azienda statunitense di proprietà del gruppo Amazon, che fornisce servizi di cloud computing su un'omonima piattaforma on demand.
	\subsection{AWS Gamelift}
        		Servizio di hosting di server specializzato nella gestione di videogiochi.
	\subsection{AWS AppSync}
	    È un servizio completamente gestito da Amazon che facilita lo sviluppo di API GraphQL gestendo le attività impegnative derivanti dalla connessione sicura a origini dati come AWS DynamoDB, Lambda e altro.
\newpage